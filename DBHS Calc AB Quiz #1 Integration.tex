\documentclass{exam}
\usepackage[utf8]{inputenc}

\begin{document}

\begin{center}
\fbox{\fbox{\parbox{5.5in}{\centering DBHS Calc AB Supplementary Quiz 1\\20 Questions, 60 minutes, 100 points\\~\\Topics covered: Basic Integration, Integration with Trigonometry, Average Value, Derivative and Integrals of Various Transcendentals, and Miscellaneous Review\\~\\Directions: Answer the following questions to the best of your ability. Show all your work. Provide exact answers when possible. If there is no solution, write DNE.}}}
\end{center}

\vspace{5mm}

\begin{questions}
\question[5] Solve: \[\int(3x^2\sin(4x^3)+2)dx\]

\vspace{\stretch{0.25}}

\question[5] Solve: \[\frac{d}{dx}(e^\pi+\frac{4}{x}-\ln(2x^2))\]
\vspace{\stretch{0.25}}

\question[5] Solve: \[\int (e^\pi+x-\frac{1}{2})dx\]
\vspace{\stretch{0.25}}

\newpage

\question[5] Solve: \[\frac{d}{dx}(7x^4-\frac{3}{x}+\ln7)\]
\vspace{\stretch{0.25}}

\question[5] Solve: \[\int (2x^2+x^\pi-\frac{17}{x})dx\]
\vspace{\stretch{0.25}}

\question[5] Solve: \[\int_{0}^{\pi}(\sin(2x)+\cos(2x))dx\]
\vspace{\stretch{0.25}}

\question[5] Solve: \[\int \tan xdx\]
\vspace{\stretch{0.25}}

\newpage

\question[5] Solve: \[\int 4\sec^2x\tan xdx\]
\vspace{\stretch{0.25}}

\question[5] Solve: \[\int \frac{-9x^2-82x-18}{x+9}dx\]
\vspace{\stretch{0.25}}

\question[5] Solve: (Hint: Use Box Method or U-Sub!)\[\int \frac{3x^2+4x}{x^3+2x^2+5}dx\]
\vspace{\stretch{0.25}}

\question[5] Solve: (Simplify completely for credit!)\[\int_{e}^{e^2}\frac{1}{x\ln(x)}dx\]
\vspace{\stretch{0.25}}

\newpage

\question[5] Solve: (Hint: Do you remember your trig identities?)\[\int_{\pi}^{e} \frac{\cos^2x}{1-\sin^2x}dx \]
\vspace{\stretch{0.25}}

\question[5] Solve: \[\int_{0}^{\pi/4} 8\sin x\cos xdx\]
\vspace{\stretch{0.25}}

\question[5] Find the average value of the function \(f(x) = 2x^2+5\) on the interval [2,4].
\vspace{\stretch{0.25}}

\question[5] Find the average value of the function \(f(x) = \cos x\) on the interval \([\frac{\pi}{6},\frac{\pi}{2}]\).
\vspace{\stretch{0.25}}

\newpage

\question[5] Solve: \[\lim_{h\rightarrow 0} \frac{3sin(\frac{\pi}{2}+h)-3}{h}\]
\vspace{\stretch{0.25}}

\question[5] Solve: \[\lim_{x\rightarrow 2} \frac{x-2}{\sqrt{x^2-4}}\]
\vspace{\stretch{0.25}}

\question[5] Find the tangent line to the function \(f(x) = 4\sin^2x + 12\) at the point where \(x = 0\).
\vspace{\stretch{0.25}}

\newpage

\question[5] If the constant acceleration of a particle traveling about a particular function is given by \(a(t)=4\), and the velocity of the same particle at \(t=0\) is \(6m/s\), and the position of the same particle at \(t = 0\) is \(2m\), find the position of the particle when \(t = 4s\). 
\vspace{\stretch{0.25}}

\question Use the following information to answer the parts. A particle travels along a continuous function \(x(t)\). 
\vspace{\stretch{0.05}}

\begin{parts}
\part [\half] The particle speeds up on the interval from \(t = 3s\) to \(t=5s\). If the velocity between this time interval is positive, what sign must the acceleration have?
\vspace{\stretch{0.1}}
\part [\half] If instead, the particle is slowing down on the interval \(t = 3s\) to \(t=5s\) while the velocity between this time interval remains positive, what sign must the acceleration have?
\vspace{\stretch{0.1}}
\part Consider the continuous function \(x(t)=t^4+4t^3+6\). on the interval \([-5,1]\).
\vspace{\stretch{0.05}}
\begin{subparts}
\subpart[1] For which values, if any, of t does the particle momentarily stop moving for an instant?
\vspace{\stretch{0.1}}
\subpart[1] For which values, if any, of t does the function have an inflection point?
\vspace{\stretch{0.1}}
\subpart[2] When does the particle reach it's leftmost position? What is this value?
\end{subparts}
\end{parts}
\end{questions}

\end{document}